\setcounter{secnumdepth}{3}

\chapter{\ifenglish Cooperative Details\else รายเอียดการไปสหกิจศึกษา\fi}


\section{\ifenglish Foundation of DevOps\else ปรับความรู้พื้นฐานของการเป็น DevOps\fi}
แนวทางในการเริ่มต้นทำงานในสายงาน DevOps จำเป็นต้องมีการศึกษาและปรับพื้นฐานความรู้ที่สำคัญเพื่อให้แน่ใจว่าพร้อมสำหรับการทำงานจริง
เนื่องจาก DevOps เป็นสายงานที่มีความใหม่และมีการพัฒนาอย่างต่อเนื่องในวงการซอฟต์แวร์ โดยหัวข้อที่ได้รับมอบหมายให้ศึกษาเพื่อเตรียมความพร้อมจะประกอบด้วย

\begin{itemize}
      \item \textbf{Docker}: เครื่องมือสำหรับการทำ Containerization เพื่อเพิ่มประสิทธิภาพในการพัฒนาและการส่งมอบซอฟต์แวร์
      \item \textbf{Kubernetes}: ระบบสำหรับการทำ Orchestration และจัดการคอนเทนเนอร์ที่ทำงานในสเกลใหญ่
      \item \textbf{Jenkins}: เครื่องมือสำหรับการทำ Continuous Integration/Continuous Deployment (CI/CD) เพื่อเพิ่มความรวดเร็วและลดข้อผิดพลาดในการพัฒนาซอฟต์แวร์
      \item \textbf{Terraform \& IaC}: เครื่องมือสำหรับการทำ Infrastructure as Code (IaC) เพื่อจัดการและปรับแต่งโครงสร้างพื้นฐานด้วยโค้ด
      \item \textbf{Monitoring Tools}: เครื่องมือที่ใช้ในการตรวจสอบและติดตามการทำงานของระบบอย่างมีประสิทธิภาพ
      \item \textbf{ELK Stack}: ระบบสำหรับการจัดการและวิเคราะห์ข้อมูล Logging เพื่อช่วยในการตรวจสอบและวิเคราะห์ปัญหาในระบบ
\end{itemize}

ทั้งนี้การศึกษาหัวข้อเหล่านี้มีระยะเวลาประมาณ 2-4 สัปดาห์ และท้ายสุดจะต้องมีการนำเสนอสิ่งที่ได้เรียนรู้
ให้กับพี่ ๆ ในทีมได้ฟังและประเมิณว่าพร้อมที่จะทำงานจริงหรือไม่ อย่างไรก็ตามรายละเอียดในหัวข้อย่อยต่าง ๆ หลังจากนี้จะเป็นการนำเสนอสิ่งที่ได้เรียนรู้และได้นำมาประยุกต์ใช้ในการทำงานจริง
ส่วนหัวข้อนอกเหนือจากที่จะกล่าวถึงก็สำคัญไม่น้อยเช่นกันแต่จะข้อนำเสนอ Documentation ที่ได้ทำสรุปการเรียนรู้มาแล้วนั้นในส่วนภาคผนวก

% \import{chapters/detail_sections}{docker.tex}
% \clearpage
% \import{chapters/detail_sections}{kube.tex}
\clearpage
\import{chapters/detail_sections}{jenkins.tex}
\clearpage
\import{chapters/detail_sections}{terraform.tex}
\clearpage

\section{\ifenglish Responsibity Task\else งานที่ได้รับมอบหมาย\fi}

\section{\ifenglish Benefit\else สวัสดิการที่ได้รับ\fi}
บริษัท SCB TechX ให้ความสำคัญกับเรื่องของสวัสดิการที่ดีให้กับพนักงาน โดยเฉพาะนักศึกษาสหกิจศึกษาที่เข้ามาทำงานในบริษัท
ถึงแม้จะไม่ได้เป็นพนักงานประจำ แต่ก็ได้รับสวัสดิการที่ดีจากบริษัทอย่างเช่น
\begin{itemize}
      \item Mac Book Pro ให้ใช้งานระหว่างระยะเวลาทำงาน
      \item อาหารเช้าและน้ำดื่มฟรีทุกวันทำงาน
      \item Account ของแหล่งเรียนรู้ออนไลน์เช่น Udemy ที่สามารถเรียน Course ไหน ๆ ก็ได้ฟรี
      \item กิจกรรมพัฒนานักศึกษาฝึกงานและสหกิจศึกษาทั้งด้าน Soft Skill และ Hard Skill ตลอดระยะเวลาทำงาน
      \item เบี้ยเลี้ยงในการทำงาน 500 บาท/วันทำงาน
\end{itemize}
ทั้งนี้ทั้งหมดหมดที่กล่าวมาเป็นเพียงส่วนหนึ่งของสวัสดิการที่ได้รับจากบริษัท SCB TechX และยังมีสวัสดิการอื่น ๆ ที่ยังไม่ได้กล่าวถึง