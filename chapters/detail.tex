% Define groups of JIRA cards
\newcommand{\supportJiraCards}{
  XDO-7387 & CBS-SunCBS | Destroy Dev02 resources \\
  XDO-7389 & CBS-SunCBS Request to add role to managed identity \\
  XDO-7394 & CBS-MCR | Provisioning Storage Account for Dev01 and QA01 \\
  XDO-7421 & CBS-SunCBS | Request to set parameter on MySQL to Off on IaC \\
  *XDO-7422 & CBS-DAP | Setup ADF to send diagnostic log to DAP log analytic workspace \\
  XDO-7429 & Require provision the existing azure manage identity to version v.1.2.1 \\
  *XDO-7444 & CBS-SunCBS Request to install teleport for MySQL flexible servers \\
  XDO-7453 & CBS MCR | Request to grant db\_datareader access to the Azure Automation Account user-managed identity \\
  XDO-7490 & CBS-SunCBS | setup Mysql parameter on QA01 \\
  XDO-7604 & CBS-SunCBS | Provisioning Infrastructure resource for Migration-Dev1 (DEV03) \\
  XDO-7686 & CBS-SunCBS | Asssement of changing Storage account ADLS Gen2 \\
  XDO-7745 & CBS-SunCBS | Change storage account from General to ADLS Gen2 on Dev01 \\
}

\newcommand{\developJiraCards}{
  XDO-7423 & POC ADF Storage Event Trigger Over SFTP
  XDO-7483 & [ADF][linked-services] Convert module from IAC next gen to xplatform multicloud \\
  XDO-7484 & [ADF][trigger] Convert module from IAC next gen to xplatform multicloud \\
  XDO-7561 & CBS-DAP | Enhance Azure Data factory linked services modules \\
  XDO-7633 & [PYMD] Develop check appconfig for Azure Databrick \\
  XDO-7737 & [CBS] Update linked service catalog to support custom sftp \\
  XDO-7742 & [CBS] Create Linked Services documentation \\
  XDO-7743 & CBS-SunCBS | Research on Entra ID with CosmosDB for PostgreSQL \\
  XDO-7744 & [CBS] | POC Dynamic create trigger via ARM Template\\
  XDO-7760 & [CBS] Make the ADF dynamic trigger document \\
}

% Create a table of JIRA cards with hyperlinks
\newcommand{\jiradata}{
  \begin{center}
    \begin{tabular}{|c|p{0.7\textwidth}|}
      \hline
      \multicolumn{1}{|c|}{\textbf{รหัส Jira Card}} & \multicolumn{1}{c|}{\textbf{หัวข้องาน}} \\
      \hline
      \multicolumn{2}{|c|}{\textbf{Support Cards}}                                         \\
      \hline
      \supportJiraCards
      \hline
      \multicolumn{2}{|c|}{\textbf{Develop Cards}}                                         \\
      \hline
      \developJiraCards
      \hline
    \end{tabular}
  \end{center}
}

% Helper command to create hyperlinks
\newcommand{\jiralink}[1]{\hyperref[desc:#1]{\texttt{#1}}}

\setcounter{secnumdepth}{3}

\chapter{\ifenglish Cooperative Details\else รายเอียดการไปสหกิจศึกษา\fi}


\section{\ifenglish Foundation of DevOps\else ปรับความรู้พื้นฐานของการเป็น DevOps\fi}
แนวทางในการเริ่มต้นทำงานในสายงาน DevOps จำเป็นต้องมีการศึกษาและปรับพื้นฐานความรู้ที่สำคัญเพื่อให้แน่ใจว่าพร้อมสำหรับการทำงานจริง
เนื่องจาก DevOps เป็นสายงานที่มีความใหม่และมีการพัฒนาอย่างต่อเนื่องในวงการซอฟต์แวร์ โดยหัวข้อที่ได้รับมอบหมายให้ศึกษาเพื่อเตรียมความพร้อมจะประกอบด้วย

\begin{itemize}
      \item \textbf{Docker}: เครื่องมือสำหรับการทำ Containerization เพื่อเพิ่มประสิทธิภาพในการพัฒนาและการส่งมอบซอฟต์แวร์
      \item \textbf{Kubernetes}: ระบบสำหรับการทำ Orchestration และจัดการคอนเทนเนอร์ที่ทำงานในสเกลใหญ่
      \item \textbf{Jenkins}: เครื่องมือสำหรับการทำ Continuous Integration/Continuous Deployment (CI/CD) เพื่อเพิ่มความรวดเร็วและลดข้อผิดพลาดในการพัฒนาซอฟต์แวร์
      \item \textbf{Terraform \& IaC}: เครื่องมือสำหรับการทำ Infrastructure as Code (IaC) เพื่อจัดการและปรับแต่งโครงสร้างพื้นฐานด้วยโค้ด
      \item \textbf{Monitoring Tools}: เครื่องมือที่ใช้ในการตรวจสอบและติดตามการทำงานของระบบอย่างมีประสิทธิภาพ
      \item \textbf{ELK Stack}: ระบบสำหรับการจัดการและวิเคราะห์ข้อมูล Logging เพื่อช่วยในการตรวจสอบและวิเคราะห์ปัญหาในระบบ
\end{itemize}

ทั้งนี้การศึกษาหัวข้อเหล่านี้มีระยะเวลาประมาณ 2-4 สัปดาห์ และท้ายสุดจะต้องมีการนำเสนอสิ่งที่ได้เรียนรู้
ให้กับพี่ ๆ ในทีมได้ฟังและประเมิณว่าพร้อมที่จะทำงานจริงหรือไม่ อย่างไรก็ตามรายละเอียดในหัวข้อย่อยต่าง ๆ หลังจากนี้จะเป็นการนำเสนอสิ่งที่ได้เรียนรู้และได้นำมาประยุกต์ใช้ในการทำงานจริง
ส่วนหัวข้อนอกเหนือจากที่จะกล่าวถึงก็สำคัญไม่น้อยเช่นกันแต่จะข้อนำเสนอ Documentation ที่ได้ทำสรุปการเรียนรู้มาแล้วนั้นในส่วนภาคผนวก

% \import{chapters/detail_sections}{docker.tex}
% \clearpage
% \import{chapters/detail_sections}{kube.tex}
\clearpage
\import{chapters/detail_sections}{jenkins.tex}
\clearpage
\import{chapters/detail_sections}{terraform.tex}
\clearpage

\section{\ifenglish TOR\else TOR\fi}เนื่องจากการมาสหกิจศึกษาจำเป็นต้องมีการประเมิณที่เข้มงวด ดังนั้นจึงได้จัดทำ TOR ขึ้นเพื่อเป็นมาตรฐานในการทำงานและประเมิณผลงานมราควรจะทำได้ดังนี้
\begin{itemize}
      \item ทำงานในลักษณะของ Task based จำนวณ 30 tasks เป็นขึ้นต่ำ เนื่องจากทีม DevOps ของ SCB Techx ไม่ได้ใช้การทำงานแบบ Agile จึงทำให้ลักษณะการจะเป็น Kanban กล่าวคือ เมื่อมีงานใหม่เข้ามาจะสามารถทำได้ทันทีโดยไม่ต้องมีการ Sprint planning ก่อน ดังนั้นเองการทำงานจึงต้องอาศัยความรอบคอบและความรวดเร็วในการทำงาน เพื่อตอบสนองความต้องการของทีม Dev และ ลูกค้าให้ได้มากที่สุด
\end{itemize}
ดังนั้นรายหัวข้อต่อไปจะเป็นการนำเสนอ Task งานที่ได้รับมอบหมายให้ทำทั้งหมด เพื่อแสดงให้เห็นว่าสามารถบรรลุเป้าหมายของ TOR ได้

\section{\ifenglish Responsibity Task\else งานที่ได้รับมอบหมาย\fi}
\jiradata


\section{\ifenglish Benefit\else สวัสดิการที่ได้รับ\fi}
บริษัท SCB TechX ให้ความสำคัญกับเรื่องของสวัสดิการที่ดีให้กับพนักงาน โดยเฉพาะนักศึกษาสหกิจศึกษาที่เข้ามาทำงานในบริษัท
ถึงแม้จะไม่ได้เป็นพนักงานประจำ แต่ก็ได้รับสวัสดิการที่ดีจากบริษัทอย่างเช่น
\begin{itemize}
      \item Mac Book Pro ให้ใช้งานระหว่างระยะเวลาทำงาน
      \item อาหารเช้าและน้ำดื่มฟรีทุกวันทำงาน
      \item Account ของแหล่งเรียนรู้ออนไลน์เช่น Udemy ที่สามารถเรียน Course ไหน ๆ ก็ได้ฟรี
      \item กิจกรรมพัฒนานักศึกษาฝึกงานและสหกิจศึกษาทั้งด้าน Soft Skill และ Hard Skill ตลอดระยะเวลาทำงาน
      \item เบี้ยเลี้ยงในการทำงาน 500 บาท/วันทำงาน
\end{itemize}
ทั้งนี้ทั้งหมดหมดที่กล่าวมาเป็นเพียงส่วนหนึ่งของสวัสดิการที่ได้รับจากบริษัท SCB TechX และยังมีสวัสดิการอื่น ๆ ที่ยังไม่ได้กล่าวถึง