\subsection{Jenkins}
Jenkins คือ Software (Tool) ตัวนึงที่เอามาใช้ทำ CI/CD \cite{cicd} เพื่อที่จะสามารถทำให้งานของ Dev \& Dev
ถูกพัฒนาและส่งมอบให้กับลูกค้าได่เร็วขึ้น โดยที่ Jenkins จะช่วยในการทำงานของการ Build, Test, Deploy
ทั้งนี้ในบริบทของการใช้งาน Jenkins ของ DevOps ในงานจริงอาจแตกต่างออกไปดังนั้นในหัวข้อนี้จะเป็นการสรุปว่า
DevOps ใช้ Jenkins ทำอะไรบ้าง และ Jenkins ช่วยในการทำงานของ DevOps อย่างไร

Jenkins จะมีหนึ่งความสามารถที่เรียกว่า Jenkins Pipeline ซึ่ง DevOps เองก็เอาความสามารถตรงนี้มาใช้ในการทำงาน
ไม่ว่าจะเป็นการ Provisioning \& Destroying resources, Deploying โดยการเขียน Script ในรูปแบบของ Jenkinsfile
ขึ้นมาเพื่อทำงานตามที่ต้องการเช่น Pipeline สำหรับ Deploy microserivce Pipeline นี้ก็จะทำงานสำหรับการ Deploy โดยเฉพาะ
โดยที่เมื่อมีการสั่งให้ทำงาน Jenkins จะทำงานตามที่เขียนไว้ใน Jenkinsfile หลังจาก Jenkins ทำงานเสร็จสิ้นก็เป็นอันว่า microserivce นั้น Deploy สำเร็จ

จะเห็นได้ว่าเมื่อเรามี Jenkins เข้ามาช่วยทำงานที่เป็น Routine ที่เราต้องอะไรเดิม ๆ ซ้ำ ๆ ทำให้เราสามารถลดเวลาในการทำงานลงได้และยังช่วยให้
Productivity ของทีมทำงานเพิ่มขึ้นอีกด้วย ทั้งจากที่กล่าวมาข้างต้นว่า DevOps ใช้ Jenkinsในการ Provisioning resource ต่าง ๆด้วยนั้นหลักการก็จะคล้าย ๆ กับการ Deploy
เพียงแต่ Script ที่ใช้สั่งการนั้นจะเปลี่ยนแปลงมห้เป็น Script สำหรับ Provisioning resource แทน ทั้งนี้ทำให้เราสามารถทำงานได้เร็วขึ้นและลดความผิดพลาดในการทำงานลงได้

\begin{minted}[frame=single,breaklines]{groovy}
    pipeline {
        agent any
        stages {
            stage('Build') {
                steps {
                    echo 'Building...'
                }
            }
            stage('Deploy') {
                steps {
                    echo 'Deploying...'
                }
            }
        }
    }
\end{minted}
