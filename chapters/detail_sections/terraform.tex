\subsection{Terraform}
Terraform คือเครื่องมือหนึ่งที่ใช้ในการสร้างและจัดการโครงสร้างพื้นฐาน (Infrastructure) ด้วยแนวคิด \textit{Infrastructure as Code} (IaC) โดยที่ Terraform นั้นมีความสามารถในการ Provision และ Manage resource ต่าง ๆ ใน cloud provider ได้หลายแห่งเช่น AWS, Azure, และ GCP รวมถึง on-premises environments อื่น ๆ ซึ่งช่วยให้การจัดการโครงสร้างพื้นฐานเป็นไปอย่างมีประสิทธิภาพและสามารถควบคุมได้ง่ายขึ้น

\section*{การใช้งาน Terraform ในบริบทของ DevOps}
ในบริบทของการใช้งานจริง Terraform มักจะถูกนำมาใช้โดยทีม DevOps เพื่อจัดการและ Provision resource ต่าง ๆ เช่น เซิร์ฟเวอร์, ฐานข้อมูล, Network configurations และอื่น ๆ รวมถึงสามารถใช้ Terraform ในการจัดการ \textit{infrastructure lifecycle} ทั้งหมดไม่ว่าจะเป็นการสร้าง, เปลี่ยนแปลง, หรือการลบ resources ได้อย่างง่ายดายและเป็นอัตโนมัติ

Terraform มีโครงสร้างที่เรียบง่ายในการใช้งาน ซึ่งประกอบด้วยการเขียน Configuration files (มักเป็นไฟล์ที่มีนามสกุล \texttt{.tf}) เพื่อกำหนด resource ที่ต้องการ ซึ่งไฟล์เหล่านี้สามารถจัดเก็บไว้ในระบบ version control (เช่น Git) เพื่อให้สามารถทำการ versioning และการทำงานร่วมกันในทีมได้

นอกจากนี้ Terraform ยังมีความสามารถในการ \textit{plan} และ \textit{preview} การเปลี่ยนแปลงที่จะเกิดขึ้นกับ infrastructure ก่อนที่จะทำการ \textit{apply} จริง ทำให้สามารถลดความเสี่ยงจากการปรับเปลี่ยนโครงสร้างพื้นฐานที่อาจทำให้เกิดปัญหาได้

\section*{ข้อสรุปของการใช้งาน Terraform โดย DevOps}
\begin{itemize}
    \item \textbf{Provisioning Resources}: ใช้ในการสร้าง resource ต่าง ๆ ใน cloud หรือ on-premises
    \item \textbf{Infrastructure as Code (IaC)}: ทำให้การจัดการ infrastructure มีความคล่องตัวและควบคุมได้ง่ายขึ้น
    \item \textbf{Automation}: ทำให้การจัดการ resource เป็นอัตโนมัติ ลดงาน manual
    \item \textbf{Version Control}: Configuration files สามารถจัดเก็บและ versioning ได้ ทำให้การทำงานร่วมกันในทีมง่ายขึ้น
    \item \textbf{Safety}: สามารถ \textit{plan} และ \textit{preview} การเปลี่ยนแปลงก่อน \textit{apply} จริงเพื่อลดความเสี่ยง
\end{itemize}